\begin{resumo}
Com o uso das redes sem fios, que disponibilizam Internet nos dispositivos móveis, a quantidade de serviços e conteúdos oferecidos aos usuários é imensa e como cada provedor de conteúdo pode apresentar atributos relacionados aos serviços oferecidos, como por exemplo: número de clientes conectados, tempo de resposta, custo e qualidade do serviço, os valores desses atributos podem ser modificados dependendo da situação do provedor. Dessa forma, o conjunto de atributos pode ser relevante para o usuário no momento da escolha do provedor de conteúdo que atenderá suas expectativas de forma satisfatória, caso não esteja sendo atendido, o usuário irá procurar outro serviço semelhante que lhe atenda da forma desejada. Assim, neste trabalho foram implementadas duas aplicações, uma aplicação servidora que realiza a escolha do serviço que melhor atende às configurações dos atributos definidas pelo usuário e que também realiza as autenticações nos serviços de forma transparente e uma aplicação para dispositivos móveis onde o usuário realiza a configuração dos atributos de acordo com suas preferências e que também é consumidora dos serviços escolhidos de forma automatizada pela aplicação servidora.
\end{resumo}	
